\documentclass[11pt, oneside]{article}   	% use "amsart" instead of "article" for AMSLaTeX format
\usepackage{geometry}                		% See geometry.pdf to learn the layout options. There are lots.
\geometry{a4paper}	                   		% ... or a4paper or a5paper or ... 
%\geometry{landscape}                		% Activate for for rotated page geometry
%\usepackage[parfill]{parskip}    		% Activate to begin paragraphs with an empty line rather than an indent
\usepackage{graphicx}				% Use pdf, png, jpg, or epsß with pdflatex; use eps in DVI mode
								% TeX will automatically convert eps --> pdf in pdflatex		
\usepackage{amssymb}

\title{Agile Development Processes \newline Post mortem report}
%\subtitle{Post mortem report}
\author{Cyrill Averbeck}
%\date{}							% Activate to display a given date or no date

\begin{document}
\maketitle

\section{Introduction}
%\section{}


In this course we worked in a group of 6 persons. 


\subsection{Processes}
\subsection{Scrum}
\subsection{Pairprogramming}
Advantage: better code, two people know the source code. Faster research. Better ideas and work.
Disadvantage: It was slower, because we were not used to this method. We could have done many more user stories if we would worked individually. This disadvantage got less important over time. However in some situations like debugging we split up and worked individually to track down the error faster.
Efficient: Not time efficient. But working is more fun than working alone.
Future: I would love to use this technique in future project, but I don't think it will be applicable in many projects. From my experience time is more often critical than quality. But I will definitely try to use as often as possible. As I mentioned I would not use this technique in project with limited time and few group members. In those projects I would prefer to use regular programming techniques 

� � �

\subsection{Unit tests/Test driven development}

\section{Conclusion}

\subsection{Notes:}
1. Which processes and practices did you use in your project?

2. Approximately, how much time was spent (in total and by each group member) on the steps/activities involved as well as for the project as a whole?

3. For each of the techniques and practices used in your project you should answer all the questions:
1. What was the advantage of this technique based on your experience in this assignment?
2. What was the disadvantage of this technique based on your experience in this assignment?
3. How efficient was the technique given the time it took to use?
4. In which situations would you use this technique in a future project?
5. In which situations would you not use this technique in a future project?
6. If you had the practice/technique in a part of the project and not the entire project, how was using it compared to not using it?

4. Compared to other projects using a more plan-driven/waterfall process what were the benefits and drawbacks with the process used in this project?

5. What worked well in how you worked in this project?


\end{document}  